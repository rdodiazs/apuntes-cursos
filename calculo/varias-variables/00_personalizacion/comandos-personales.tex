%=========================================
%
%    Comandos personalizados usados en
%       los apuntes de este curso.
%
%=========================================

%% Vectores y matrices.
\newcommand{\vecmat}[1]{\mathbf{#1}}                          % Vectores o matrices en negrita en math mode.
\newcommand{\unitvec}[1]{\vecmat{\hat{#1}}}                   % Vectores unitarios.
\newcommand{\overvec}[1]{\overrightarrow{#1}}                 % Vector como segmento orientado.
\newcommand{\proy}[2]{\text{proy}_{\vecmat{#2}}{\vecmat{#1}}} % Proyección vectorial.
\newcommand{\invmat}[1]{\vecmat{#1}^{-1}}                     % Inversa de una matriz.
\newcommand{\transmat}[1]{\vecmat{#1}^{T}}                    % Transpuesta de una matriz.
\newcommand{\Adj}[0]{\text{Adj}}                              % Matriz adjunta.


%% Conjuntos numéricos.
\newcommand{\R}[0]{\mathbb{R}}                                % Símbolo conjunto de los números reales.
\newcommand{\N}[0]{\mathbb{N}}                                % Símbolo conjunto de los números naturales.


%% Límite de función bivariada.
\newcommand{\limbiv}[1]{\lim_{(x, \ y) \to (#1)}}


%% Derivadas.
% Derivada ordinaria (una variable).
\newcommand{\deriv}[2][]{\frac{d #1}{d #2}} % Notación de Leibniz.

%% Derivada parcial (notación con el símbolo "parcial").
\newcommand{\derivpar}[2][]{\frac{\partial #1}{\partial #2}}
\newcommand{\pderivpar}[3]{\left.\derivpar{#1} #2 \right|_{#3}}                  % Derivada parcial en un punto.
\newcommand{\nderivpar}[3][]{\frac{\partial^{#2} #1}{\partial #3^{#2}}}          % n-ésimo orden con respecto a una misma variable.
\newcommand{\nbiderivpar}[4][]{\frac{\partial^{#2} #1}{\partial #3 \partial #4}} % n-ésimo orden con respecto a dos variables distintas.


%% Misceláneos.
% Contador (counter) para enumerar los ejemplos.
\newcounter{ejem}
\setcounter{ejem}{1}
\newcommand{\ejemplo}{\textbf{Ejemplo \theejem.}\stepcounter{ejem} }

% Atajo para la palabra `solución` en negrita.
\newcommand{\solucion}{\textbf{Solución.} }
