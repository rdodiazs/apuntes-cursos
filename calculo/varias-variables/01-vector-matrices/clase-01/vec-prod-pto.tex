\documentclass[12pt]{article}
\usepackage[margin=1in]{geometry}
\usepackage[utf8]{inputenc}
\usepackage[spanish]{babel}
\usepackage{parskip}
\usepackage{setspace}
\usepackage{amsmath, amssymb}
\usepackage{tikz}
\usepackage{hyperref} % Siempre debe ir al final.

% Opciones de Paquetes.
\decimalpoint          % {babel}
\onehalfspacing        % {setspace}
\usetikzlibrary{babel} % {tikz}: Para que tikz no conflictue con {babel} con figuras como "->".

% Encabezado.
\title{Clase 1. Vectores y el Producto Punto.}
\author{MIT 18.02: Multivariable Calculus.}
\date{}


\begin{document}

% Comandos personalizados.
\newcommand{\vecmat}[1]{\mathbf{#1}}                          % Vectores o matrices en negrita en math mode.
\newcommand{\overvec}[1]{\overrightarrow{#1}}                 % Vector como segmento orientado.
\newcommand{\proy}[2]{\text{proy}_{\vecmat{#2}}{\vecmat{#1}}} % Proyección vectorial.
\newcommand{\R}[0]{\mathbb{R}}                                % Símbolo conjunto de los números reales.
\newcommand{\N}[0]{\mathbb{N}}                                % Símbolo conjunto de los números naturales.

\maketitle

\begin{abstract}
\noindent En esta primera clase revisaremos el concepto de \textbf{vector}. Veremos las formas de representarlo y las operaciones básicas que podemos realizar entre ellos. También profundizaremos en una de estas últimas conocida como \textbf{producto punto}.
\end{abstract}

\section{Vectores.}

En este curso, un \textbf{vector} será entendido como una cantidad que posee una \textbf{magnitud} y una \textbf{dirección}. Si consiste \textbf{solo de una magnitud}, entonces será llamado como \textbf{escalar}.

Los vectores se denotan con una letra minúscula escrita en negrita (por ej., $\vecmat{a}$) o con una flecha arriba de ella (por ej., $\vec{a}$). Acá optaremos por la primera.

A nivel geométrico, los vectores son \textbf{segmentos orientados}\footnote{Un segmento es una parte de una recta acotado por dos puntos extremos distintos.} que están acotados por un \textbf{punto inicial} y otro \textbf{terminal} que indica su \textbf{dirección}. La \textbf{magnitud} es dada por el \textbf{largo} de esta figura.

\begin{figure}[hbt!]
\centering

\begin{tikzpicture}
% Líneas de ayuda.
%\draw[color = lightgray] (-8, -2) grid (8, 2);
%\draw[color = gray] (-8, 0) -- (8, 0);
%\draw[color = gray] (0, -2) -- (0, 2);

% Recta.
\draw[color = red] (-5, 0) -- (5, 0) node [above] {$L$};

% Magnitud del vector.
% Sobre loops en tikz: https://latexdraw.com/how-to-use-the-foreach-loop-in-latex
\foreach \i in {-2.5, 2.5} {
  \draw[line width = 0.2mm, color = gray] (\i, 1) -- (\i, 1.45);  % Líneas verticales.
}
\foreach \j \k in {-2.5 / -1.35, 2.5 / 1.35} {
  \draw[line width = 0.2mm, color = gray] (\j, 1.2) -- (\k, 1.2); % Líneas horizontales.
}
\node[font = \small] at (0, 1.2) {Magnitud de $\vecmat{a}$};

% Puntos terminales.
\draw[fill = black] (-2.5, 0) circle (0.1) node [above] {$A$} node [below, align = center, font = \small] {Punto \\ inicial};
\draw[fill = black] (2.5, 0) circle (0.1) node [above] {$B$} node [below, align = center, font = \small] {Punto \\ terminal};

% Vector.
\draw[-stealth, line width = 0.7mm] (-2.37, 0) -- node [above] {$\vecmat{a} = \overvec{AB}$} (2.37, 0);
\end{tikzpicture}

\caption{Vector $\vecmat{a}$ representado como un segmento orientado, $\overvec{AB}$, de una recta $L$.}

\end{figure}


\section{Representación de un vector.}

Es posible representar a un vector mediante sus componentes, usando vectores unitarios o a partir de la relación entre su magnitud y dirección. A continuación los revisamos.

\subsection{Forma componente.}

Un vector puede ser expresado algebraicamente como la \textbf{posición de su punto terminal} en un \textbf{sistema de coordenadas}.

Consideremos un vector $\vecmat{a}$ en un plano cartesiano, con su punto inicial $A$ ubicado en $(x_{0}, \ y_{0})$ y su punto terminal en $(x_{1}, \ y_{1})$.

\begin{figure}[hbt!]
\centering

\begin{tikzpicture}
% Líneas de ayuda.
%\draw[color = lightgray] (-2, -1) grid (14, 6); % Celdas.
%\draw[color = gray] (0, -1) -- (0, 6);          % Línea vertical.
%\draw[color = gray] (-2, 0) -- (14, 0);         % Línea horizontal.

% Ejes de coordenadas del plano cartesiano.
\draw[-latex, line width = 0.3mm] (0, -0.3) -- (0, 4.4) node [left] {$y$}; % Eje vertical.
\draw[-latex, line width = 0.3mm] (-0.3, 0) -- (5, 0) node [below] {$x$};  % Eje horizontal.

% Etiquetas de los ejes de coordenadas.
\foreach \i \j \l in {1/0/1, 4/1/3.5} {
  \node at (\i, -0.3) {$x_{\j}$};
  \node at (-0.3, \l) {$y_{\j}$};
}

% Delimitación de coordenadas.
\draw[style = dashed, color = gray] (1, 0) -- (1, 1) -- (0, 1);
\draw[style = dashed, color = gray] (4, 0) -- (4, 3.5) -- (0, 3.5);

% Vector.
\draw[-stealth, line width = 0.6mm] (1, 1) -- node [above left] {$\vecmat{a}$} (3.91, 3.44);

% Puntos terminales.
\draw[fill = black] (1, 1) circle (0.1) node[below right] {$A(x_{0}, \ y_{0})$};
\draw[fill = black] (4, 3.5) circle (0.1) node [right] {$B(x_{1}, \ y_{1})$};

\end{tikzpicture}

\end{figure}

Para dos constantes $a_{1}$ y $a_{2}$, podemos definir las coordenadas del punto $B(x_{1}, \ y_{1})$ como:
\[
  x_{1} = x_{0} + a_{1} \qquad \qquad y_{1} = y_{0} + a_{2}
\]
Es decir, $B$ puede ser entendido como un punto desplazado desde $A$ hasta su ubicación donde el vector $\vecmat{a}$ indica su trayectoria en forma de segmento orientado.

Al despejar a las constantes $a_{1}$ y $a_{2}$ en $x_{1}$ e $y_{1}$, obtenemos los \textbf{componentes} del vector $\vecmat{a}$.
\[
  a_{1} = x_{1} - x_{0} \qquad \qquad a_{2} = y_{1} - y_{0}
\]
Así, la representación \textbf{forma componente} de un vector es aquella que indica su posición en un sistema de coordenadas como la distancia de su punto terminal con respecto al inicial.
\[
  \vecmat{a} = \langle a_{1}, \ a_{2} \rangle = \langle x_{1} - x_{0}, \ y_{1} - y_{0} \rangle
\]
Una consecuencia de esta representación es que, independiente de la ubicación de \textbf{dos o más vectores} en un sistema de coordenadas, estos \textbf{serán iguales} si \textbf{tienen la misma magnitud} y están en la \textbf{misma dirección}.

Cuando se representa a un vector en forma componente, comúnmente se usa el \textbf{origen} como punto inicial. Estos reciben el nombre de \textbf{vector de posición} o \textbf{vector estándar}.
\[
  \vecmat{b} = \langle b_{1}, \ b_{2} \rangle = \langle x_{1} - 0, \ y_{1} - 0 \rangle = \langle x_{1}, \ y_{1} \rangle
\]

\begin{figure}[hbt!]
\centering

\begin{tikzpicture}
% Líneas de ayuda.
%\draw[color = lightgray] (-2, -1) grid (14, 6); % Celdas.
%\draw[color = gray] (0, -1) -- (0, 6);          % Línea vertical.
%\draw[color = gray] (-2, 0) -- (14, 0);         % Línea horizontal.

% Ejes de coordenadas del plano cartesiano.
\draw[-latex, line width = 0.3mm] (0, -0.3) -- (0, 4) node [left] {$y$}; % Eje vertical.
\draw[-latex, line width = 0.3mm] (-0.3, 0) -- (4, 0) node [below] {$x$};  % Eje horizontal.

% Etiquetas de los ejes de coordenadas.
\node at (3, -0.3) {$x_{1}$};
\node at (-0.3, 3) {$y_{1}$};

% Delimitación coordenadas.
\draw[style = dashed, color = gray] (3, 0) -- (3, 3) -- (0, 3);

% Vector.
\draw[-stealth, line width = 0.6mm] (0, 0) -- node [above left] {$\vecmat{b}$} (2.96, 2.92);

% Puntos terminales.
\draw[fill = black] (0, 0) circle (0.1) node[below left] {$C(0, \ 0)$};
\draw[fill = black] (3, 3) circle (0.1) node[right] {$B(x_{1}, \ y_{1})$};

\end{tikzpicture}

\caption{Vector de posición $\vecmat{b} = \langle b_{1}, \ b_{2} \rangle$.}

\end{figure}

La \textbf{cantidad de componentes} de un vector está determinado por la \textbf{dimensión} del sistema de coordenadas en el que estemos trabajando. En ese sentido, en uno de tres con valores reales, o $\R^{3}$, se denotaría como:
\[
  \vecmat{a} = \langle a_{1}, \ a_{2}, \ a_{3} \rangle
             = \langle x_{1} - x_{0}, \ y_{1} - y_{0}, \ z_{1} - z_{0} \rangle;
             \text{ para } \vecmat{a} \in \R^{3}
\]
Y para uno de $n$ componentes, con $n \in \N$:
\[
  \vecmat{a} = \langle a_{1}, \ a_{2}, \ \ldots, \ a_{n} \rangle; \text{ para } \vecmat{a} \in \R^{n}
\]

\subsubsection{Operaciones básicas con vectores.}

La forma componente de un vector permite realizar operaciones básicas como la adición vectorial y la multiplicación escalar.

Si $\vecmat{a}$ y $\vecmat{b}$ son dos vectores de \textbf{igual dimensión}, entonces la \textbf{suma} entre ellos resulta en el \textbf{vector}:
\[
  \vecmat{a} + \vecmat{b} = \langle a_{1} + b_{1}, \ a_{2} + b_{2}, \ \ldots, \ a_{n} + b_{n} \rangle
\]
Una interpretación geométrica de la adición vectorial es que corresponde al vector que cierra el triángulo que se forma al unir el punto inicial de uno de ellos con el terminal del otro.

\newpage

\begin{figure}[hbt!]
\centering

\begin{tikzpicture}
% Líneas de ayuda.
%\draw[color = lightgray] (-2, -0.5) grid (14, 4);
%\draw[color = gray] (0, -0.5) -- (0, 4);
%\draw[color = gray] (-2, 0) -- (14, 0);

% Vectores.
\draw[-stealth, line width = 0.5mm, color = red!85] (2, 0) -- node[left] {$\vecmat{a} +\vecmat{b}$} (2.97, 2.9); % vector a + b.
\draw[-stealth, line width = 0.5mm] (2, 0) -- node[right] {$\vecmat{a}$} (4, 1.5); % vector a.
\draw[-stealth, line width = 0.5mm] (4, 1.5) -- node[above right] {$\vecmat{b}$} (3, 3); % vector b.
\end{tikzpicture}

\end{figure}

Otra interpretación de la adición vectorial es ser entendida como la diagonal de un paralelogramo formado entre los dos vectores que la originan.

\begin{figure}[hbt!]
\centering

\begin{tikzpicture}
% Líneas de ayuda.
%\draw[color = lightgray] (-2, -0.5) grid (14, 4);
%\draw[color = gray] (0, -0.5) -- (0, 4);
%\draw[color = gray] (-2, 0) -- (14, 0);

% Vectores.
\draw[-stealth, line width = 0.5mm, color = red!85] (1, 0) -- node [above, rotate = 50] {$\vecmat{a} + \vecmat{b}$} (3.17, 2.95);
\draw[-stealth, line width = 0.5mm] (1, 0) -- node[left] {$\vecmat{a}$} (1.3, 2);
\draw[-stealth, line width = 0.5mm] (1, 0) -- node[below right] {$\vecmat{b}$} (3, 1.2);
\draw[style = dashed, line width = 0.3mm] (3, 1.2) -- (3.2, 3) -- (1.3, 2);
\end{tikzpicture}

\end{figure}

Sea $k$ un número. El vector que proviene del producto entre este valor y un vector $\vecmat{a}$ se conoce como \textbf{multiplicación escalar}.
\[
  k \cdot \vecmat{a} = \langle k \cdot a_{1}, \ k \cdot a_{2}, \ \ldots, \ k \cdot a_{n} \rangle
\]
Como se indica en su nombre, el efecto geométrico de esta operación es escalar al vector $\vecmat{a}$ mediante $k$. Por otra parte, si el signo de $k$ es distinto al de $\vecmat{a}$, entonces $k$ también cambia la dirección de $\vecmat{a}$.

\begin{figure}[hbt!]
\centering

\begin{tikzpicture}
% Líneas de ayuda.
%\draw[color = lightgray] (-2, -0.5) grid (14,4);
%\draw[color = gray] (0, -0.5) -- (0, 4);
%\draw[color = gray] (-2, 0) -- (14, 0);

% Vectores.
\draw[-stealth, line width = 0.5mm] (0, 1) -- node[left] {$\vecmat{a}$} (1.3, 3);
\draw[-stealth, line width = 0.5mm] (1.7, 0.4) -- node[left] {$k \ \vecmat{a}$} (3.5, 3.3);
\draw[-stealth, line width = 0.5mm] (4.1, 1) -- node[left] {$\displaystyle \frac{1}{k} \ \vecmat{a}$} (4.9, 2.3);
\draw[stealth-, line width = 0.5mm] (6, 1) -- node[left] {$(-k) \ \vecmat{a}$} (7.3, 3);
\end{tikzpicture}

\end{figure}

La \textbf{sustracción} entre dos vectores $\vecmat{a}$ y $\vecmat{b}$ se puede entender como la suma vectorial con uno de ellos siendo escalado por $(-1)$.
\[
  \vecmat{a} - \vecmat{b} = \vecmat{a} + (-\vecmat{b}) = \langle a_{1} - b_{1}, \ a_{2} - b_{2}, \ \ldots, \ a_{n} - b_{n} \rangle
\]

\newpage

\begin{figure}[hbt!]
\centering

\begin{tikzpicture}
% Líneas de ayuda.
%\draw[color = lightgray] (-2, -0.5) grid (14,4);
%\draw[color = gray] (0, -0.5) -- (0, 4);
%\draw[color = gray] (-2, 0) -- (14, 0);

% Vectores.
\draw[-stealth, line width = 0.5mm, color = red!85] (3, 1.5) -- node[right] {$\vecmat{a} - \vecmat{b}$} (1.8, 3);
\draw[-stealth, line width = 0.5mm] (1, 1) -- node[below] {$\vecmat{a}$} (3, 1.5);
\draw[stealth-, line width = 0.5mm] (1, 1) -- node[left] {$-\vecmat{b}$} (1.8, 3);
\end{tikzpicture}

\end{figure}

Sean tres vectores $\vecmat{a}, \ \vecmat{b}, \ \vecmat{c} \in \R^{n}$, $k$ y $w$ dos escalares; y $\vecmat{0}$ el \textbf{vector cero}, cuyos componentes y magnitud son iguales a cero. En la siguiente tabla se resumen las principales propiedades de las operaciones entre vectores.

\begin{table}[hbt!]
\centering

\begin{tabular}{c|c}
Suma de Vectores & Multiplicación Escalar \\
\hline
$\vecmat{a} + \vecmat{b} = \vecmat{b} + \vecmat{a}$ & $k(\vecmat{a} + \vecmat{b}) = k\vecmat{a} + k\vecmat{b}$ \\
$\vecmat{a} + (\vecmat{b} + \vecmat{c}) = (\vecmat{a} + \vecmat{b}) + \vecmat{c}$ & $(k + w)\vecmat{a} = k\vecmat{a} + w\vecmat{a}$ \\
$\vecmat{a} + \vecmat{0} = \vecmat{a}$ & $(kw) \vecmat{a} = k(w\vecmat{a}) = w(k\vecmat{a})$ \\
$\vecmat{a} + (-\vecmat{a}) = \vecmat{0}$ & $1\vecmat{a} = \vecmat{a}$ \\
 & $0\vecmat{a} = \vecmat{0}$ \\
 & $c\vecmat{0} = \vecmat{0}$
\end{tabular}

\caption{Propiedades básicas de las operaciones entre vectores.}

\end{table}

\subsubsection{Magnitud de un vector.}

Los componentes de un vector también permiten obtener el valor de su magnitud. Veámoslo con $\vecmat{a}$.

Si unimos las coordenadas de los puntos inicial y terminal de $\vecmat{a}$ con segmentos discontinuos, obtenemos un triángulo rectángulo $ACB$ con catetos de longitud igual a sus componentes.

\begin{figure}[hbt!]
\centering

\begin{tikzpicture}
% Líneas de ayuda.
%\draw[color = lightgray] (-2, -1) grid (14, 6); % Celdas.
%\draw[color = gray] (0, -1) -- (0, 6);          % Línea vertical.
%\draw[color = gray] (-2, 0) -- (14, 0);         % Línea horizontal.

% Ejes de coordenadas del plano cartesiano.
\draw[-latex, line width = 0.3mm] (0, -0.3) -- (0, 4.4) node [left] {$y$}; % Eje vertical.
\draw[-latex, line width = 0.3mm] (-0.3, 0) -- (7, 0) node [below] {$x$};  % Eje horizontal.

% Etiquetas de los ejes de coordenadas.
\foreach \i \j \l in {1/0/1, 5/1/3.5} {
  \node at (\i, -0.3) {$x_{\j}$};
  \node at (-0.3, \l) {$y_{\j}$};
}

% Vector.
\draw[-stealth, line width = 0.7mm] (1.1, 1.07) -- (4.9, 3.43);

% Unión de coordenadas.
\foreach \i in {1, 3.5} {
  \draw[style = dashed] (0, \i) -- (5, \i);   % Líneas horizontales.
}
\node[below] at (3, 1) {$a_{1} = x_{1} - x_{0}$};

\foreach \j in {1, 5} {
  \draw[style = dashed] (\j, 0) -- (\j, 3.5); % Líneas verticales.
}
\node[right] at (5, 2) {$a_{2} = y_{1} - y_{0}$};

\node[below right] at (5, 1) {$C$};

\draw[fill = gray, color = gray] (5, 1) rectangle (4.7, 1.3); % Ángulo recto.

% Puntos terminales.
\draw[fill = black] (1, 1) circle (0.1) node[below left] {$A$};
\draw[fill = black] (5, 3.5) circle (0.1) node [right] {$\vecmat{a} = \langle a_{1}, \ a_{2} \rangle$} node [above] {$B$};

\end{tikzpicture}

\end{figure}

Como el largo del segmento orientado equivale a la \textbf{magnitud} de $\vecmat{a}$, también denotado como $||\vecmat{a}||$, entonces podemos aplicar el teorema de Pitágoras para obtener su valor.
\[
  ||\vecmat{a}|| = \sqrt{a_{1}^{2} + a_{2}^{2}}
\]
Es posible generalizar el cálculo de $||\vecmat{a}||$ para cualquier $\vecmat{a} \in \R^{n}$ como:
\[
  ||\vecmat{a}|| = \sqrt{\sum_{i = 1}^{n} a_{i}^{2}} = \sqrt{a_{1}^{2} + a_{2}^{2} + \ldots + a_{n}^{2}}
\]
El valor de $||\vecmat{a}||$ también se conoce como \textbf{norma euclidiana} (o norma\footnote{También es expresada como norma $l^{2}$, $2$-norma o norma cuadrática y es denotada como $|| \cdot ||_{2}$.}) y es una función $|| \cdot ||: \R \to \R_{0}^{+}$.\footnote{El símbolo $\R_{0}^{+}$ representa a los números reales no negativos (i.e, mayores o iguales a cero).} Se cataloga como ``euclidiana'' porque devuelve el valor de la distancia euclidiana del punto terminal de $\vecmat{a}$ con respecto al inicial, que es lo que acabamos de calcular.

También es posible calcular la norma de otros objetos tales como matrices (que estudiaremos más adelante) o tensores (que no estudiaremos). En este curso lo aplicaremos solo a vectores.

\subsection{Mediante vectores unitarios.}

Un vector también puede ser representado como la \textbf{adición vectorial entre vectores unitarios escalados por sus componentes}.
\[
  \vecmat{a} = a_{1} \vecmat{\hat{e}}_{1} + a_{2} \vecmat{\hat{e}}_{2} + \ldots + a_{n} \vecmat{\hat{e}}_{n}
             = \sum_{i = 1}^{n} a_{i} \vecmat{\hat{e}}_{i}
\]
donde los $\vecmat{\hat{e}}_{i}$ son vectores unitarios.

Los \textbf{vectores unitarios} son todos aquellos de \textbf{magnitud igual a} $1$. Para representar a otro vector, se suelen usar \textbf{vectores unitarios estándar}, que se caracterizan por ser \textbf{paralelos} a los ejes de un sistema cartesiano y \textbf{ortogonales} (i.e, perpendiculares) entre ellos.

\begin{figure}[hbt!]
\centering

\begin{tikzpicture}
% Líneas de ayuda.
%\draw[color = lightgray] (-2, -0.5) grid (14, 5);
%\draw[color = gray] (0, -0.5) -- (0, 5);
%\draw[color = gray] (-2, 0) -- (14, 0);

% Ejes de coordenadas.
\draw[-stealth, line width = 0.3mm] (0, -0.5) -- (0, 4.2) node[left] {$y$};
\draw[-stealth, line width = 0.3mm] (-0.5, 0) -- (5, 0) node [below] {$x$};

% Vectores.
%% Vector a.
\draw[-stealth, line width = 0.7mm] (1, 0.7) -- node[above left] {$\vecmat{a}$} (4, 3.7);

%% Vector unitario y escalado horizontal.
\draw[-stealth, line width = 0.5mm, color = blue!75] (1, 0.7) -- (4, 0.7) node[below, font = \small] {$a_{1} \vecmat{\hat{e}}_{1}$};
\draw[-stealth, line width = 0.5mm, color = red!85] (1, 0.7) -- node[below, font = \small] {$\vecmat{\hat{e}}_{1}$}(2, 0.7);

%% Vector unitario y escalado vertical.
\draw[-stealth, line width = 0.5mm, color = blue!75] (4, 0.7) -- (4, 3.7) node[right, font = \small] {$a_{2} \vecmat{\hat{e}}_{2}$};
\draw[-stealth, line width = 0.5mm, color = red!85] (4, 0.7) -- node[right, font = \small] {$\vecmat{\hat{e}}_{2}$} (4, 1.7);
\end{tikzpicture}

\caption{Vector $\vecmat{a}$ representado con vectores unitarios estándar $\vecmat{\hat{e}}_{1}$ y $\vecmat{\hat{e}}_{2}$.}

\end{figure}

También es posible transformar un vector $\vecmat{a}$ en uno unitario. Este proceso se conoce como \textbf{normalización vectorial} y consiste en multiplicarlo por el recíproco de su magnitud.
\[
  \vecmat{\hat{a}} = \frac{1}{||\vecmat{a}||} \cdot \vecmat{a}; \text{ con } ||\vecmat{\hat{a}}|| = 1
\]

\subsection{Usando su magnitud y dirección.}

Una tercera forma de representar a un vector es a partir de la información que entregan su \textbf{dirección y magnitud}.

La \textbf{dirección} de un vector es dada por el \textbf{ángulo} $\angle \theta$ que se forma entre éste y una recta horizontal $L$ que pasa por su punto de origen.

\begin{figure}[hbt!]
\centering

\begin{tikzpicture}
% Líneas de ayuda.
%\draw[color = lightgray] (0, 0) grid (10, 4);

% Recta.
\draw (0, 1) -- (6.5, 1) node [above] {$L$};
\node at (3.3, 1.3) {$\theta$};

%Vector.
\draw[-stealth, line width = 0.6mm] (2.5, 1) -- node[above left] {$\vecmat{a}$} (5, 3.5);
\end{tikzpicture}

\end{figure}

Si trazamos un segmento vertical desde el punto terminal de $\vecmat{a}$ hasta $L$, podemos formar un triángulo rectángulo $ACB$.

\begin{figure}[hbt!]
\centering

\begin{tikzpicture}
% Líneas de ayuda.
%\draw[color = lightgray] (0, 0) grid (10, 4);

% Recta.
\draw (0, 1) -- (6.5, 1) node [above] {$L$};
\node at (3.3, 1.3) {$\theta$};

% Segmento vertical.
\draw[style = dashed, color = gray] (5, 3.5) -- (5, 1) node [below, color = black] {$C$};
\draw[fill = gray, color = gray] (5, 1) rectangle (4.7, 1.3);

%Vector.
\draw[-stealth, line width = 0.6mm] (2.5, 1) node[below] {$A$} -- node[above left] {$\vecmat{a}$} (5, 3.5) node [right] {$B$};
\end{tikzpicture}

\end{figure}

Suponga que $\vecmat{a} \in \R^{2}$. Entonces $\overline{AC} = a_{1}$, $\overline{BC} = a_{2}$ y $\overline{AB} = ||\vecmat{a}||$. Usemos esta información para calcular las funciones seno y coseno de $\theta$.
\[
  \sin(\theta) = \frac{a_{2}}{||\vecmat{a}||} \qquad \qquad \cos(\theta) = \frac{a_{1}}{||\vecmat{a}||}
\]
Al despejar a $a_{1}$ y $a_{2}$ en las ecuaciones de arriba, se obtienen los componentes de $\vecmat{a}$ en términos de su dirección y magnitud.
\[
  a_{1} = ||\vecmat{a}|| \cdot \cos(\theta) \qquad \qquad a_{2} = ||\vecmat{a}|| \cdot \sin(\theta)
\]
Por lo tanto,
\[
  \vecmat{a} = \langle ||\vecmat{a}|| \cdot \cos(\theta), \ ||\vecmat{a}|| \cdot \sin(\theta) \rangle
             = [||\vecmat{a}|| \cdot \cos(\theta)] \vecmat{\hat{e}}_{1} + [||\vecmat{a}|| \cdot \sin(\theta)] \vecmat{\hat{e}}_{2}
\]
Una consecuencia de esta representación de $\vecmat{a}$ es que:
\[
  \theta = \arccos\left(\frac{a_{1}}{||\vecmat{a}||}\right) = \arcsin\left(\frac{a_{2}}{||\vecmat{a}||}\right)
\]


\section{Producto punto.}

Además de las vistas en la sección 2.1.1, otra operación importante entre vectores es el \textbf{producto punto}.

\subsection{Definición algebraica.}

Sean $\vecmat{a}, \ \vecmat{b} \ \in \R^{n}$. El producto punto entre ambos vectores es un \textbf{escalar} definido como:
\[
  \vecmat{a} \cdot \vecmat{b} = a_{1} b_{1} + a_{2} b_{2} + \ldots + a_{n} b_{n} = \sum_{i = 1}^{n} a_{i} b_{i}
\]
Mediante el producto punto se pueden obtener las siguientes propiedades, con $\vecmat{a}, \ \vecmat{b}, \ \vecmat{c} \in \R^{n}$ y $k \in \R$.

\begin{itemize}
\item[a)] $\vecmat{a} \cdot \vecmat{b} = \vecmat{b} \cdot \vecmat{a}$
\item[b)] $(k \vecmat{a}) \cdot \vecmat{b} = \vecmat{a} \cdot (k \vecmat{b}) = k (\vecmat{a} \cdot \vecmat{b})$
\item[c)] $\vecmat{a} \cdot (\vecmat{b} + \vecmat{c}) = \vecmat{a} \cdot \vecmat{b} + \vecmat{a} \cdot \vecmat{c}$
\item[d)] $\vecmat{0} \cdot \vecmat{a} = 0$
\item[e)] $\vecmat{a} \cdot \vecmat{a} = ||\vecmat{a}||^{2}$
\end{itemize}

De la propiedad e) se puede concluir que:
\[
  ||\vecmat{a}|| = \sqrt{\vecmat{a} \cdot \vecmat{a}}
\]

\subsection{Definición geométrica.}

El producto punto entre dos vectores también entrega información del ángulo que se forma entre ellos al unirlos en sus puntos iniciales.

En particular, la definición geométrica del producto punto entre dos vectores $\vecmat{a}, \ \vecmat{b} \in \R^{n}$ es:
\[
  \vecmat{a} \cdot \vecmat{b} = ||\vecmat{a}|| \ ||\vecmat{b}|| \ \cos(\theta)
\]
donde $\theta$ es el ángulo que se forma entre ellos al unirlos en sus puntos iniciales.

\textbf{Demostración.} Tracemos a los vectores $\vecmat{a}$, $\vecmat{b}$ y $\vecmat{a} - \vecmat{b}$ para formar un triángulo. También incluyamos al $\angle \theta$ que se forma entre $\vecmat{a}$ y $\vecmat{b}$.

\begin{figure}[hbt!]
\centering

\begin{tikzpicture}
% Líneas de ayuda.
%\draw[color = lightgray] (0, 0) grid (12, 4);

% Vectores.
\draw[-stealth, line width = 0.6mm] (2, 1) -- node[above left] {$\vecmat{b}$} (3.2, 3.3);
\draw[-stealth, line width = 0.6mm] (2, 1) -- node[below] {$\vecmat{a}$} (5, 1.2);
\draw[-stealth, line width = 0.6mm] (3.2, 3.3) -- node[right] {$\vecmat{a} - \vecmat{b}$} (4.97, 1.22);
\node at (2.4, 1.3) {$\theta$};
\end{tikzpicture}

\end{figure}

Luego, calculemos el lado $||\vecmat{a} - \vecmat{b}||$ mediante la \textbf{ley del coseno}.
\[
  ||\vecmat{a} - \vecmat{b}||^{2} = ||\vecmat{a}||^{2} + ||\vecmat{b}||^{2} - 2 ||\vecmat{a}|| \ ||\vecmat{b}|| \cos(\theta)
\]
Al aplicar las propiedades e), c) y a) del producto punto, obtenemos lo siguiente:
\begin{align*}
  (\vecmat{a} - \vecmat{b}) \cdot (\vecmat{a} - \vecmat{b}) &= (\vecmat{a} \cdot \vecmat{a}) + (\vecmat{b} \cdot \vecmat{b})
                                                              - 2 ||\vecmat{a}|| \ ||\vecmat{b}|| \cos(\theta) \\
  (\vecmat{a} \cdot \vecmat{a}) - (\vecmat{a} \cdot \vecmat{b}) -
  (\vecmat{b} \cdot \vecmat{a}) + (\vecmat{b} \cdot \vecmat{b}) &= (\vecmat{a} \cdot \vecmat{a}) + (\vecmat{b} \cdot \vecmat{b})
                                                                    - 2 ||\vecmat{a}|| \ ||\vecmat{b}|| \cos(\theta) \\
                                -2(\vecmat{a} \cdot \vecmat{b}) &= -2 ||\vecmat{a}|| \ ||\vecmat{b}|| \cos(\theta) \\
                                    \vecmat{a} \cdot \vecmat{b} &= ||\vecmat{a}|| \ ||\vecmat{b}|| \cos(\theta) \quad \text{(Q. E. D)}
\end{align*}
De la definición geométrica del producto punto podemos sacar dos conclusiones. La primera es que:
\[
  \cos(\theta) = \frac{\vecmat{a} \cdot \vecmat{b}}{||\vecmat{a}|| \ ||\vecmat{b}||}
  \qquad \text{y} \qquad
  \theta = \arccos\left(\frac{\vecmat{a} \cdot \vecmat{b}}{||\vecmat{a}|| \ ||\vecmat{b}||}\right)
\]
La segunda conclusión corresponde a que si $\theta = \frac{\pi}{2}$, entonces:
\[
  \vecmat{a} \cdot \vecmat{b} = 0; \text{ ya que } \cos\left(\frac{\pi}{2}\right) = 0
\]
Es decir, dos vectores son \textbf{ortogonales} si su producto punto es \textbf{igual a cero}.

Observemos también que:
\[
  \vecmat{a} \cdot \vecmat{b} > 0 \iff 0 < \theta < \frac{\pi}{2}
  \qquad \text{y} \qquad
  \vecmat{a} \cdot \vecmat{b} < 0 \iff \frac{\pi}{2} < \theta < \pi
\]
porque
\[
  \cos(\theta) > 0 \iff 0 < \theta < \frac{\pi}{2}
  \qquad \text{y} \qquad
  \cos(\theta) < 0 \iff \frac{\pi}{2} < \theta < \pi
\]
\textbf{Ejemplo 1.} Calcule el $\angle \theta$ del siguiente triángulo $QPR$ ubicado en el espacio cartesiano\footnote{El ``espacio cartesiano'' es el sistema de coordenadas cartesiano en tres dimensiones.}.

\begin{figure}[hbt!]
\centering

\begin{tikzpicture}
% Líneas de ayuda.
%\draw[color = lightgray] (-5, -2) grid (5, 3);
%\draw[color = gray] (0, -2) -- (0, 3);
%\draw[color = gray] (-5, 0) -- (5, 0);

% Ejes de coordenadas.
\draw[-latex, line width = 0.3mm, color = black!80] (0, 0) -- (-2.1, -1.2) node[color = black, below] {$x$};
\draw[-latex, line width = 0.3mm, color = black!80] (0, 0) -- (2.4, 0) node[color = black, below] {$y$};
\draw[-latex, line width = 0.3mm, color = black!80] (0, 0) -- (0, 2.6) node[color = black, left] {$z$};
\node at (-0.2, 0.15) {$0$};

% Triángulo.
\draw[line width = 0.5mm, color = teal] (-1, -0.6) -- (1, 0) -- (0, 2) -- cycle;
\node at (-0.65, -0.3) {$\theta$};

% Puntos de coordenadas.
\draw[color = teal, fill = teal] (-1, -0.6) circle (0.1) node[below right, color = black] {$P(1, \ 0, \ 0)$};
\draw[color = teal, fill = teal] (1, 0) circle (0.1) node[above right, color = black] {$Q(0, \ 1, \ 0)$};
\draw[color = teal, fill = teal] (0, 2) circle (0.1) node[left, color = black] {$R(0, \ 0, \ 2)$};
\end{tikzpicture}

\end{figure}

\textbf{Solución.} Definamos a los vectores
\begin{align*}
  \overvec{PQ} &= \langle 0 - 1, \ 1 - 0, \ 0 \rangle = \langle -1, \ 1, \ 0 \rangle \\
  \overvec{PR} &= \langle 0 - 1, \ 0, \ 2 - 0 \rangle = \langle -1, \ 0, \ 2 \rangle
\end{align*}
Luego, calculemos las normas de $\overvec{PQ}$ y $\overvec{PR}$.
\begin{align*}
  ||\overvec{PQ}|| &= \sqrt{(-1)^{2} + 1^{2} + 0^{2}} = \sqrt{2} \\
  ||\overvec{PR}|| &= \sqrt{(-1)^{2} + 0^{2} + 2^{2}} = \sqrt{5}
\end{align*}
Por lo tanto,
\begin{align*}
  \theta = \arccos\left(\frac{\overvec{PQ} \cdot \overvec{PR}}{||\overvec{PQ}|| \ ||\overvec{PR}||}\right)
         = \arccos\left(\frac{\langle -1, \ 1, \ 0 \rangle \cdot \langle -1, \ 0, \ 2 \rangle}{(\sqrt{2}) (\sqrt{5})}\right)
         \approx 75.56^{\circ}
\end{align*}
\textbf{Ejemplo 2.} Explique por qué la ecuación $x + 2y + 3z = 0$ genera un plano.

\textbf{Solución.} Se puede observar que esta ecuación genera un plano al expresar su lado izquierdo como el producto punto entre dos vectores $\vecmat{a} = \langle 1, \ 2, \ 3 \rangle$ y $\vecmat{x} = \langle x, \ y, \ z \rangle$.
\[
  \vecmat{a} \cdot \vecmat{x} = 0
\]
Todas las soluciones de $x + 2y + 3z = 0$ contenidas en $\vecmat{x}$ serán ortogonales a sus coeficientes que están en $\vecmat{a}$. En consecuencia, al graficarlas se formará un plano.

\subsection{Dirección y magnitud mediante el producto punto.}

En la sección 2.3, para representar a un vector usando su ángulo de dirección y su magnitud se usó una recta horizontal que cruzaba su punto inicial. El producto punto entrega otro método para expresarlo con la misma información.

Considere un vector de posición $\vecmat{a} \in \R^{2}$, a sus unitarios estándar $\vecmat{\hat{e}}_{1} = \langle 1, \ 0 \rangle$ y $\vecmat{\hat{e}}_{2} = \langle 0, \ 1 \rangle$; y a los ángulos $\alpha_{1}$ y $\alpha_{2}$ que se forman entre $\vecmat{a}$ y los $\vecmat{\hat{e}}_{i}$.

\begin{figure}[hbt!]
\centering

\begin{tikzpicture}
% Líneas de ayuda.
%\draw[color = lightgray] (-5, -2) grid (5, 4);
%\draw[color = gray] (0, -2) -- (0, 4);
%\draw[color = gray] (-5, 0) -- (5, 0);

% Ejes de coordenadas.
\draw[-latex, line width = 0.3mm] (0, 0) -- (3.2, 0) node[below] {$x$};
\draw[-latex, line width = 0.3mm] (0, 0) -- (0, 3.2) node[left] {$y$};
\node at (-0.2, 0) {$0$};

% Vectores.
\draw[-stealth, line width = 0.6mm] (0, 0) -- (2.2, 1.4) node[right] {$\vecmat{a} = \langle a_{1}, \ a_{2} \rangle$};
\draw[-stealth, line width = 0.6mm, color = red!75] (0, 0) -- node[below] {$\vecmat{\hat{e}}_{1}$} (1.5, 0) node[below, color = black] {$1$};
\draw[-stealth, line width = 0.6mm, color = red!75] (0, 0) -- node[left] {$\vecmat{\hat{e}}_{2}$} (0, 1.4) node[left, color = black] {$1$};

% Ángulos.
\node[font = \small] at (0.8, 0.23) {$\alpha_{1}$};
\node[font = \small] at (0.3, 0.6) {$\alpha_{2}$};

%Arcos.
\draw[-stealth, line width = 0.2mm] (1.1, 0) arc (0:30:1.2);
\draw[-stealth, line width = 0.2mm] (1, 0.68) arc [start angle = 0, end angle = 130, x radius = 0.6, y radius = 0.5];
\end{tikzpicture}

\end{figure}

Luego, calculemos los productos punto entre $\vecmat{a}$ y los $\vecmat{\hat{e}}_{i}$.
\begin{align*}
\vecmat{a} \cdot \vecmat{\hat{e}}_{1} &= ||\vecmat{a}|| \ ||\vecmat{\hat{e}}_{1}|| \ \cos(\alpha_{1}) &
\vecmat{a} \cdot \vecmat{\hat{e}}_{2} &= ||\vecmat{a}|| \ ||\vecmat{\hat{e}}_{2}|| \ \cos(\alpha_{2}) \\
a_{1} + 0 &= ||\vecmat{a}|| \ 1 \ \cos(\alpha_{1}) &
0 + a_{2} &= ||\vecmat{a}|| \ 1 \ \cos(\alpha_{2}) \\
a_{1} &= ||\vecmat{a}|| \ \cos(\alpha_{1}) &
a_{2} &= ||\vecmat{a}|| \ \cos(\alpha_{2})
\end{align*}
De esta manera, podemos representar al vector $\vecmat{a}$ como
\[
  \vecmat{a} = \langle ||\vecmat{a}|| \ \cos(\alpha_{1}), \ ||\vecmat{a}|| \ \cos(\alpha_{2}) \rangle
\]
Si $\vecmat{a} \in \R^{n}$, entonces:
\[
  \vecmat{a} = \langle ||\vecmat{a}|| \ \cos(\alpha_{1}), \ ||\vecmat{a}|| \ \cos(\alpha_{2}),
                       \ \ldots, \ ||\vecmat{a}|| \ \cos(\alpha_{n}) \rangle
\]
De las igualdades de $a_{1}, \ \ldots, \ a_{n}$ también se puede obtener que:
\[
  \cos(\alpha_{1}) = \frac{a_{1}}{||\vecmat{a}||}, \quad \cos(\alpha_{2}) = \frac{a_{2}}{||\vecmat{a}||},
                     \quad \ldots, \quad \cos(\alpha_{n}) = \frac{a_{n}}{||\vecmat{a}||}
\]
Mediante estos cosenos surge la siguiente identidad:
\[
  \sum_{i = 1}^{n} \cos^{2}(\alpha_{i}) = 1
\]
Se puede demostrar al reemplazar los cosenos en la suma.
\begin{align*}
  \sum_{i = 1}^{n} \cos^{2}(\alpha_{i}) = \sum_{i = 1}^{n} \left(\frac{a_{i}}{||\vecmat{a}||}\right)^{2}
                                        = \frac{1}{||\vecmat{a}||^{2}} \ \sum_{i = 1}^{n} a_{i}^{2}
                                        = \frac{\vecmat{a} \cdot \vecmat{a}}{\vecmat{a} \cdot \vecmat{a}}
                                        = 1
\end{align*}

\subsection{Proyecciones vectoriales ortogonales.}

En ocasiones se necesita buscar un vector que sea similar a otro en términos de su magnitud o dirección (o ambos). Una opción es \textbf{proyectarlo ortogonalmente} sobre otro objeto. Cuando éste es un vector, podemos usar el producto punto para realizar esta tarea.

Suponga que queremos proyectar un vector $\vecmat{a}$ sobre otro $\vecmat{b}$. Para ello, los unimos en sus puntos iniciales y trazamos un segmento ortogonal a $\vecmat{b}$, $\overline{DE}$, desde el punto terminal de $\vecmat{a}$.

\begin{figure}[hbt!]
\centering

\begin{tikzpicture}
% Líneas de ayuda.
%\draw[color = lightgray] (0, 0) grid (10, 4);

% Segmento perpendicular.
\draw[style = dashed] (4, 3.7) node[right, color = black] {$D$} -- (4.17, 1.3) node[below, color = black] {$E$};
\draw[color = gray, fill = gray] (4.15, 1.58) -- (3.85, 1.545) -- (3.87, 1.25) -- (4.17, 1.27) -- cycle;

% Vectores.
\draw[-stealth, line width = 0.5mm] (1, 1) -- (4, 3.7) node[above left] {$\vecmat{a}$};
\draw[-stealth, line width = 0.5mm] (1, 1) node[left] {$C$} -- (5, 1.3) node[below] {$\vecmat{b}$};
\node at (1.8, 1.35) {$\theta$};
\end{tikzpicture}

\end{figure}

En ese sentido, la proyección de $\vecmat{a}$ sobre $\vecmat{b}$ será el vector $\overvec{CE}$, que denotaremos como $\proy{a}{b}$.

\begin{figure}[hbt!]
\centering

\begin{tikzpicture}
% Líneas de ayuda.
%\draw[color = lightgray] (0, 0) grid (10, 4);

% Segmento perpendicular.
\draw[style = dashed] (4, 3.7) node[right, color = black] {$D$} -- (4.17, 1.3) node[below, color = black] {$E$};
\draw[color = gray, fill = gray] (4.15, 1.58) -- (3.85, 1.545) -- (3.87, 1.25) -- (4.17, 1.27) -- cycle;

% Vectores.
\draw[-stealth, line width = 0.5mm] (1, 1) node[below] {$C$} -- (4, 3.7) node[above left] {$\vecmat{a}$};
\draw[-stealth, line width = 0.5mm] (1, 1) -- (5, 1.3) node[below] {$\vecmat{b}$};
\node at (1.8, 1.35) {$\theta$};
\draw[-stealth, line width = 0.5mm, color = red!85] (1, 1) -- node[below] {$\proy{a}{b}$} (4.17, 1.24); % Proyección de a sobre b.
\end{tikzpicture}

\end{figure}

Al igual que con cualquier otro vector, necesitamos conocer la magnitud y dirección de $\proy{a}{b}$. El primer valor se puede obtener mediante el $\cos(\theta)$ del triángulo rectángulo ECD que vemos en la gráfica de arriba.
\begin{align*}
                   \cos(\theta) &= \frac{||\proy{a}{b}||}{||\vecmat{a}||} \\
   \therefore \ ||\proy{a}{b}|| &= ||\vecmat{a}|| \ \cos(\theta)
\end{align*}
El escalar $||\proy{a}{b}||$ también se conoce como ``componente de $\vecmat{a}$ sobre $\vecmat{b}$'' o ``proyección escalar''.

Luego, necesitamos que $\proy{a}{b}$ esté en dirección de $\vecmat{b}$. Como está sobre este último y ambos comienzan en el mismo lugar, podemos usar el vector normalizado de $\vecmat{b}$ para que sea escalado por $||\proy{a}{b}||$.
\[
  (||\proy{a}{b}||) \ \frac{\vecmat{b}}{||\vecmat{b}||} = (||\vecmat{a}|| \ \cos(\theta)) \ \frac{\vecmat{b}}{||\vecmat{b}||}
\]
Por lo tanto, la proyección de $\vecmat{a}$ sobre $\vecmat{b}$ se define como:
\[
  \proy{a}{b} = \left(\frac{||\vecmat{a}|| \ \cos(\theta)}{||\vecmat{b}||}\right) \ \vecmat{b}  
\]
Esta fórmula es útil solo si conocemos a $\theta$. A partir del \textbf{producto punto} entre $\vecmat{a}$ y $\vecmat{b}$ podemos obtener otra que no depende de este valor.
\[
  \vecmat{a} \cdot \vecmat{b} = ||\vecmat{a}|| \ ||\vecmat{b}|| \ \cos(\theta)
\]
Anteriormente vimos que $||\proy{a}{b}|| = ||\vecmat{a}|| \ \cos(\theta)$. Por lo tanto,
\[
  \vecmat{a} \cdot \vecmat{b} = ||\proy{a}{b}|| \ ||\vecmat{b}||
\]
Al multiplicar por el recíproco de $||\vecmat{b}||$ en la igualdad de arriba, obtenemos la proyección escalar de $\vecmat{a}$ sobre $\vecmat{b}$ sin tener que considerar a $\theta$.
\[
  ||\proy{a}{b}|| = \frac{\vecmat{a} \cdot \vecmat{b}}{||\vecmat{b}||}
\]
Finalmente, al multiplicar por el vector normalizado de $\vecmat{b}$ se concluye que:
\[
  \proy{a}{b} = \left(\frac{\vecmat{a} \cdot \vecmat{b}}{||\vecmat{b}||}\right) \ \frac{\vecmat{b}}{||\vecmat{b}||}
              = \left(\frac{\vecmat{a} \cdot \vecmat{b}}{||\vecmat{b}||^{2}}\right) \ \vecmat{b}
\]
Es posible que $||\proy{a}{b}|| < 0$. El motivo se da por el valor que tome $\theta$. Como vimos en la sección 3.2, si $\frac{\pi}{2} < \theta < \pi$, entonces $\cos(\theta) < 0$. La interpretación geométrica de aquello es que la $\proy{a}{b}$ está sentido contrario a $\vecmat{b}$.

\begin{figure}[hbt!]
\centering

\begin{tikzpicture}
%Líneas de ayuda.
%\draw[color = lightgray] (0, 0) grid (10, 4);

% Segmento ortogonal.
\draw[style = dashed, color = gray, line width = 0.3mm] (2.5, 3) -- (2.5, 1);
\draw[fill = gray, color = gray] (2.5, 1) rectangle (2.8, 1.3);

%Vectores.
\draw[-stealth, line width = 0.5mm] (5, 1) node[above] {$\theta$} -- (2.5, 3) node[above]{$\vecmat{a}$};
\draw[-stealth, line width = 0.5mm] (5, 1) -- (8, 1) node[above] {$\vecmat{b}$};
\draw[-stealth, line width = 0.5mm, color = red!80] (5, 1) -- (2.5, 1) node[below] {$\proy{a}{b}$};
\end{tikzpicture}

\caption{Vector $\proy{a}{b}$ con $\frac{\pi}{2} < \theta < \pi$.}

\end{figure}

\end{document}
