\documentclass[12pt]{article}
\usepackage[utf8]{inputenc}
\usepackage[margin=1in]{geometry}
\usepackage[spanish]{babel}
\usepackage{parskip}
\usepackage{setspace}
\usepackage{amsmath}
\usepackage{hyperref}

% Opciones de paquetes.
\decimalpoint % {babel}: coma decimal como punto.
\onehalfspacing % {setspace}: interlineado

% Encabezado.
\title{Clase 30. Integración por Partes y Fórmula de Reducción.}
\author{MIT 18.01: Single Variable Calculus.}
\date{}

\begin{document}
\maketitle

\begin{abstract}
\noindent La \textbf{integración por partes} es otra de las técnicas usadas para obtener integrales que no pueden ser resueltas directamente por el Teorema Fundamental del Cálculo. En esta clase estudiaremos cómo usarla y, a partir de ella, conoceremos \textbf{fórmulas de reducción} que también son útiles para resolver integrales de forma más rápida.
\end{abstract}

\section{Integración por Partes.}

En ocasiones, integrales del producto entre dos funciones,
\[
  \int f(x)g(x) dx,
\]
no pueden ser resueltas directamente. Sin embargo, existe una técnica de integración que nos permite separarlas, usando la derivada de una de ellas e integramos solo la segunda. Ésta recibe el nombre de \textbf{Integración por Partes}.

La integración por partes podemos entenderla como la ``antiderivada'' de la regla del producto de las derivadas\footnote{Una relación similar a la que hay entre el método de sustitución (directa o trigonométrica) de las integrales y la regla de la cadena de las derivadas.} ya que, como ahora veremos, su fórmula se obtiene a partir de ella.

Sean $u = f(x)$ y $v = g(x)$. Por la regla del producto, la derivada de su multiplicación corresponde a:
\[
  (uv)' = u'v + uv'
\]
Luego, despejemos a $uv'$ de la ecuación.
\[
  uv' = (uv)' - u'v
\]
Y, posteriormente, calculemos la integral en la ecuación.
\[
  \int uv' dx = \int [(uv)' - u'v] dx = \int (uv)' dx - \int u'v dx
\]
Por el Teorema Fundamental del Cálculo podemos establecer que $\int (uv)' dx = uv$. Así,
\[
  \int uv' dx = uv - \int u'v dx
\]
es la \textbf{fórmula} que nos permite aplicar la técnica de \textbf{integración por partes}.

Para las \textbf{integrales definidas} con límites $a$ y $b$, donde $a < b$, la fórmula de la integración por partes es la siguiente:
\[
  \int_{a}^{b} uv' dx = [uv]_{a}^{b} - \int_{a}^{b} u'v dx
\]

\textbf{Ejemplo 1.} Calcule la siguiente integral:
\[
  \int \ln(x) dx
\]
\textbf{Solución.} Esta integral no es integrable directamente a partir del TFC y tampoco con los métodos de integración que hemos estudiado en clases anterior. No obstante, podemos expresarla como el producto entre dos funciones:
\[
  \int \ln(x)dx = \int (\ln(x) \cdot 1) dx
\]
En ese sentido, tenemos la opción para ver si podemos encontrar funciones que son más fáciles de integrar a partir de la técnica de integración por partes.

Como vimos en la fórmula de la integración por partes, tenemos que elegir a cuál de las dos funciones, $\ln(x)$ o $1$, consideraremos como una derivada. Para este caso, la mejor opción es la función constante $1$. Por lo tanto, establezcamos que:
\[
  u = \ln(x) \qquad \text{ y } \qquad v' = 1
\]
Esto implica que:
\[
  u' = \frac{1}{x} \qquad \text{ y } \qquad v = x
\]
Así, usando el método de integración por partes, podemos reescribir la integral inicial como:
\[
  \int \ln(x) dx = x \ln(x) - \int \frac{1}{x} x dx = x \ln(x) - \int 1 dx = x \ln(x) - x + C
\]

\textbf{Ejemplo 2.} Calcule la integral
\[
  \int (\ln(x))^{2} dx
\]
\textbf{Solución.} Esta integral también es posible resolverla por medio de la técnica de integración por partes y de igual modo a como lo hicimos en el ejemplo anterior, estableciendo que:
\[
  u = (\ln(x))^{2}, \ u' = \frac{2 (\ln(x))}{x}, \ v' = 1, \ v = x
\]
Por lo tanto,
\[
  \int (\ln(x))^{2} dx = x (\ln(x))^{2} - \int \frac{2 (\ln(x))}{x} x dx = x (\ln(x))^{2} - 2 \int \ln(x) dx
\]
En el Ejemplo 1 vimos que $\int \ln(x)dx = x \ln(x) - x$. En consecuencia:
\[
  \int (\ln(x))^{2} dx =  x (\ln(x))^{2} - 2 (x \ln(x) - x)
\]

\subsection{Fórmulas de Reducción a partir de la Integración por Partes.}

En el ejemplo anterior, la antiderivada de $\int (\ln(x))^{2} dx$ se construyó a partir de $\int \ln(x) dx$. Esta no es una coincidencia. Efectivamente, es un patrón existente en $\ln(x)$ que está determinado por su exponente y, con ella, es posible obtener una \textbf{Fórmula de Reducción}.

Las fórmulas de reducción son aquellas que están definidas por funciones con términos anteriores a ella. A partir de la técnica de integración por partes, podemos encontrar este tipo de expresiones en algunas integrales.

Por ejemplo, resolvamos la integral
\[
  \int (\ln(x))^{n}dx
\]
Utilizando el método de integración por partes, podemos establecer que:
\[
  u = (\ln(x))^{n} \qquad \text{ y } \qquad v' = 1
\]
implicando que
\[
  u' = \frac{n \cdot (\ln(x))^{n - 1}}{x} \qquad \text{ y } \qquad v = x
\]
Por lo tanto
\[
  \int (\ln(x))^{n}dx = x (\ln(x))^{n} - n \int (\ln(x))^{n - 1} dx
\]
Podemos simplificar esta expresión definiendo que $F_{n}(x) = \int (\ln(x))^{n} dx$.
\[
  F_{n}(x) = x \cdot (\ln(x))^{n} - n \cdot F_{n - 1}(x)
\]
Fórmulas de reducción como la del $\int (\ln(x))^{n} dx$ son útiles porque permiten resolver integrales de manera más rápida.

A continuación evaluamos a $F_{n}(x)$ para $n = {1, \ 2}$.
\begin{align*}
  F_{1}(x) &= x (\ln(x))^{1} - 1 \cdot \int (\ln(x))^{1 - 1} dx & F_{2}(x) &= x (\ln(x))^{2} - 2 \cdot \int (\ln(x))^{2 - 1} dx \\
           &= x \ln(x) - x + C                                  &          &= x (\ln(x))^{2} - 2 (x \ln(x) - x) + C
\end{align*}
Como vemos, nos ahorramos el uso de técnicas de integración al encontrar la fórmula de reducción de $\int (\ln(x))^{n} dx$.

\textbf{Ejemplo 3.} Busque la fórmula de reducción de
\[
  \int x^{n} e^{x} dx
\]
\textbf{Solución.} Usaremos el método de integración por partes para obtener la fórmula de reducción de este ejemplo.

Como señalamos antes, una fórmula de reducción se caracteriza por funciones que son de término menor (anterior) a la de la función original. En ese sentido, la mejor opción es que
\[
  u = x^{n} \qquad \text{ y } \qquad v' = e^{x},
\]
porque al derivar $u$ obtendremos un factor cuyo término irá reduciéndose a medida que $n$ aumenta. En cambio, si lo hacemos con $e^{x}$ no veremos algún progreso en dicho propósito, porque $d/dx \ e^{x} = e^{x}$. Es decir, lo anterior implica que:
\[
  u' = nx^{n - 1} \qquad \text{ y } \qquad v = e^{x}
\]
Aplicando el método de integración por partes, obtenemos la fórmula de reducción para la integral de este ejemplo:
\[
  \int x^{n} e^{x} dx = x^{n} e^{x} - n \int x^{n - 1} e^{x} dx,
\]
la cual podemos simplificar estableciendo que $G_{n}(x) = \int x^{n} e^{x} dx$.
\[
  G_{n}(x) = x^{n} e^{x} - n \cdot G_{n - 1}(x)
\]
\textbf{Ejemplo 4.} Calcule las fórmulas de reducción de
\[
  \text{(a) } \int x^{n} \sin(x) dx \qquad \text{ y } \qquad \text{(b) } \int x^{n} \cos(x) dx
\]
\textbf{Solución.} Ambas fórmulas se encuentran de igual modo a como lo hicimos en el ejemplo anterior.

Tanto para (a) como para (b), definamos que $u = x^{n}$, lo que significa que $u' = n x^{n - 1}$.

Luego, centrándonos en (a), digamos que $v' = \sin(x)$. Esto implica que $v = - \cos(x)$. Por lo tanto, su fórmula de reducción es:
\[
  \int x^{n} \sin(x) dx = n \int x^{n - 1} \cos(x) dx - x^{n} \cos(x)
\]
En cuanto a (b), definamos que $v' = \cos(x)$ y que, por consiguiente, $v = \sin(x)$. En consecuencia, la fórmula de reducción de esta integral la encontramos de la siguiente manera:
\[
  \int x^{n} \cos(x) dx = x^{n} \sin(x) - n \int x^{n - 1} \sin(x) dx
\]
A continuación resumimos las fórmulas de reducción para las integrales que hemos visto hasta ahora.
\begin{align*}
  &\text{(1) } \int (\ln(x))^{n}dx = x (\ln(x))^{n} - n \int (\ln(x))^{n - 1} dx \\
  &\text{(2) } \int x^{n} e^{x} dx = x^{n} e^{x} - n \int x^{n - 1} e^{x} dx \\
  &\text{(3) } \int x^{n} \sin(x) dx = n \int x^{n - 1} \cos(x) dx - x^{n} \cos(x) \\
  &\text{(4) } \int x^{n} \cos(x) dx = x^{n} \sin(x) - n \int x^{n - 1} \sin(x) dx
\end{align*}

\end{document}
