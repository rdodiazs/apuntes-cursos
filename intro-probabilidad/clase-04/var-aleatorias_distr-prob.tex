\documentclass[12pt]{article}
\usepackage[margin=1in]{geometry}
\usepackage[utf8]{inputenc}
\usepackage[spanish]{babel}
\usepackage{parskip}
\usepackage{setspace}
\usepackage{amsmath, amssymb}
\usepackage{hyperref} % Siempre debe ir al final.

% Opciones de Paquetes.
\decimalpoint   % {babel}
\onehalfspacing % {setspace}

% Encabezado.
\title{Clase 4: Variables Aleatorias y Distribución de Probabilidad.}
\author{Harvard Statistics 110: Probability}
\date{}


\begin{document}

\maketitle

\begin{abstract}
\noindent En esta clase se realiza una introducción a los conceptos de variable aleatoria y de distribución de probabilidad, con mayor detalle en el primero. Ambos son fundamentales para comprender los próximos temas que se estudiarán en este curso.
\end{abstract}

% Simbolo del conjunto de los numeros reales
\newcommand{\R}{\mathbb{R}}

\section{Variables Aleatorias.}

Consideremos el experimento de lanzar cuatro veces una moneda no sesgada de dos lados. Su espacio muestral $S$ consiste de $2^{4} = 16$ elementos o resultados muestrales. Un evento $E \subseteq S$ sería obtener tres caras, donde:
\[
  E = \{CSCC, \ SCCC, \ CCSC, \ CCCS\}
\]
En este caso, $|E| = 4$, pero si decidimos lanzar ocho veces la misma moneda, entonces\footnote{Uso la siguiente fórmula para calcular la cantidad de permutaciones de un multiconjunto, conocido también como coeficiente multinomial:
\[
  \binom{n}{n_{1} \ n_{2} \ \ldots \ n_{r}} = \frac{n!}{n_{1}! \cdot n_{2}! \cdot \ldots \cdot n_{r}!};
  \quad \text{donde} \sum_{i = 1}^{r} n_{i} = n
\]
Ver más en Larsen y Marx (2018). \textit{An Introduction to Mathematical Statistics and Its Applications}. Pp 77.}:
\[
  |E| = \frac{8!}{3! \cdot 5!} = 56
\]
Es decir, el conjunto $E$ consistirá de $56$ secuencias con tres caras y cinco sellos.

Es claro que si realizamos más lanzamientos de la moneda, el evento $E$ se hará cada vez más complejo de expresar. Sin embargo, si solo nos interesan las tres caras, las secuencias pasan a ser irrelevantes porque en todas ellas se cumple aquello. Para reducir toda esa tarea, es más conveniente definir una función $X$ que \textbf{resuma} numéricamente esa cantidad.

Volvamos al experimento de lanzar cuatro veces una moneda. Para mayor legibilidad, establezcamos que $E = \{CSCC, \ SCCC, \ CCSC, \ CCCS\} = \{e_{1}, \ e_{2}, \ e_{3}, \ e_{4} \}$. Si $X$ es una función de la cantidad de caras de los resultados del espacio muestral $S$, entonces:
\[
  X(e_{1}) = X(e_{2}) = X(e_{3}) = X(e_{4}) = 3
\]
En probabilidades, $X$ recibe el nombre de \textbf{Variable Aleatoria} y es definida como una \textbf{función} $X \colon S \mapsto \R$, con $S$ siendo el espacio muestral de un experimento aleatorio. En ese sentido, cada imagen de ella se expresa como $x = X(s)$. Por lo tanto, su \textbf{rango} será el conjunto $\{x \colon X(s) = x; \ \forall s \in S\}$.

Como una variable aleatoria $X$ es una regla que transforma a cada resultado muestral $s \in S$ en un número $x \in \R$, su \textbf{rango} pasa a ser un \textbf{espacio muestral resumido} del mismo experimento.

Por ejemplo, en el experimento de lanzar ocho veces una moneda, su espacio muestral consiste de $2^{8} = 256$ resultados muestrales. Si $X$ es una variable aleatoria de la cantidad de caras obtenidas, su rango será el conjunto $\{0, \ 1, \ 2, \ 3, \ \ldots, \ 8\}$. También es un espacio muestral (cada número es un resultado del experimento), pero asociado solo a $X$.

Como a veces es difícil expresar un resultado muestral\footnote{Sea en una cadena de caracteres, en notación de conjuntos, etc.} $s \in S$, se denotará al resultado de una variable aleatoria $X$ como $X = x$ en vez de $X(s) = x$. De igual modo se hará para otros operadores de relación, tales como $>$, $<$, etc.


\section{Distribución de probabilidad.}

Puesto que las variables aleatorias están definidas en el espacio muestral de un experimento aleatorio, un paso lógico a seguir es calcular la probabilidad de que suceda ese valor o qué tan posible es que ocurra dentro de un intervalo (cerrado o abierto) de números. En otras palabras, a toda variable aleatoria se le puede asociar una función (o medida) de probabilidad.

Un concepto que surge del vínculo entre las variables aleatorias y sus funciones de probabilidad es el de \textbf{distribución de probabilidad}. Corresponde a una \textbf{colección} de \textbf{todas las probabilidades asociadas a cada valor que puede tomar una variable aleatoria}. En ese sentido, dicho listado describe el \textbf{comportamiento} probabilístico de esta última función.

Producto del desarrollo teórico de las probabilidades y de su aplicación en muchas áreas del conocimiento, se han podido generalizar distintas distribuciones de probabilidad las que, debido a la relevancia en cuanto a sus características, han llegado a ser nombradas. Algunos ejemplos son las distribuciones binomial o la de Gauss.

Dependiendo del hecho o fenómeno que se quiera representar mediante una variable aleatoria, se va a querer asumir (al menos hipotéticamente) que sigue una distribución de probabilidad determinada. Por ejemplo, si señalamos que $X$ es una variable aleatoria que es descrita por una distribución binomial, entonces lo denotamos como:
\[
  X \sim \text{Binom}(n, \ p)
\]
En la expresión Binom$(n, \ p)$, tanto $n$ como $p$ son los parámetros que definen a la distribución binomial y que debemos conocer de algún modo\footnote{O al menos estimarlas, que es algo que se estudia más profundamente en Estadística.} para afirmar que, efectivamente, la variable aleatoria $X$ sigue dicha distribución.

En el estudio de las probabilidades, las variables aleatorias han sido divididas en dos grandes grupos\footnote{También se identifica un tercer grupo de variables aleatorias conocidas como \textbf{mixtas}, que se caracterizan por ser tanto discretas como continuas.}:

\begin{enumerate}
\item Variables aleatorias discretas.
\item Variables aleatorias continuas.
\end{enumerate}

En las siguientes clases estudiaremos ambos grupos de variables aleatorias, comenzando con las discretas. Al hacerlo, también profundizaremos en sus funciones y distribuciones de probabilidad más relevantes.

\end{document}